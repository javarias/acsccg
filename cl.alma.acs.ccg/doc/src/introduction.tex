\section{INTRODUCTION} \label{sec:intro}

\subsection{Purpose}
The creation of components based over Alma Common Software (ACS), can be a very
repetitive task, for example the configuration of the CDB, definition of the IDL
files and the classes creation, always is pretty much the same development
process. Here is when the Model Driven Development (MDD) becomes a powerfull
tool for the code generation, in this case, based on defined templates for UML
class diagrams, part of this work is already done in the thesis $[1]$ of Nicolas
Troncoso.\\
\\
In order to improve development time, code refactoring and the software
engineering (at pattern desings level) for the component development over ACS,
will continue the work already done by Nicolas Troncoso proposed in his
thesis $[1]$, please check thesis in Referenced Documents section in this
document.\\
\\
The purpose of this document is to explain the new features, the new
development,how the generator works and how the generation 'generates' the code
from UML model/metamodel/meta-metamodels, as well as providing guidelines for
the future development and extensions of the component code generator.

\subsection{Scope}
The content of the document is for developers, software engineers and managers,
to allow them to use in a easy and good way the component generator for ACS,
to create in a fast way code ready for the implementation.\\
\\
It is assumed that the reader has a good command in the creation of components
for ACS and understand the functioning of ACS and knows Java OOP paradigm. 

\subsection{Reference Documents}
$[1]$ ACS Component Code Generation Framework\\
\hspace*{0.45cm} https://csrg.inf.utfsm.cl/twiki4/pub/ACS/AlmaTheses/thesis-ntroncos09.pdf\\
\\
$[2]$ Eclipse Modeling Framework Documents\\
\hspace*{0.45cm} http://www.eclipse.org/modeling/emf/docs/\\
\\
$[3]$ Agile Model Driven Development with UML 2\\
\hspace*{0.45cm} Cambridge University Press, 2004 ISBN\#: 0-521-54018-6\\ 
\\
$[4]$ Open Architecture Ware Reference\\
\hspace*{0.45cm} http://www.openarchitectureware.org/pub/documentation/4.3.1/

\newpage

\subsection{Abbreviations and Acronyms}
\begin{itemize}
	\item ACS : Alma Common Software.
	\item EMF : Eclipse Modeling Framework, a framework of Eclipse to create
	plug-ins, Eclipse Applications for code generation base in Model Driven.
	\item NC : Notification Channels of ACS.
	\item oAW : Open Architecture Ware, a free open architecture for code generation.
	\item XMI : XML Metadata Interchange, standard for exchanging metadata information via Extensible Markup Language (XML).
	\item UML : Unified Modeling Language, is a standardized general-purpose modeling language in the field of software engineering.
	\item MDD: Model Driven Development
	\item OOP : Object Oriented Programming
\end{itemize}

\subsection{Stylistic Conventions}

The following styles are used:

{\bf bold}\\
\hspace{5cm} In the text, to highlight words.\\
%\hspace{5cm} In the text, for commands, filenames, pre/suffixes
%as they have to be typed.\\

{\it italic}\\
\hspace{5cm} In the text, for parts that have to be substituted
with the real content before typing. Also used to highlight words or section
names.\\

{\tt teletype}\\
\hspace{5cm} In the text, for commands, filenames, pre/suffixes
as they have to be typed. Also used for file content examples.\\

\verb+<name>+\\
\hspace{5cm} In examples, for parts that have to be substituted
with the real content before typing.\\

\verb+<<stereotype>>+\\
\hspace{5cm} Stereotypes used in the code generator.


\newpage
