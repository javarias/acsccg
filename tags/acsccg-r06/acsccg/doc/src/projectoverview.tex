\section{PROJECT OVERVIEW}
\subsection{Previous Work}
The work done before this project is related to the thesis$[1]$ of Nicolas
Troncoso, which gives the very beginning to the ACS Code Generation based on UML
metamodels, this project will cover and continue the work already done
, proposed in the thesis as a 'Future Work', like the implementation of
inheritance, more complex models, separation of implemented and generated
classes and have a full implemented code generator for ACS.

\subsection{Project Objectives}
The main goal/objective in the project is have a full Standalone Java Component
Code Generator for Alma Common Software under Eclipse Modeling Framework
umbrella, this, is divided in specific objectives:

\begin{itemize}
\item Project based in Open Architecture Ware 5:\\
All the project has to be based on Open Architecture Ware 5, now under the EMF
umbrella.
 
\item Managment of complex UML Models:\\
In complex UML models, the use of stereotypes that allow to filter the models
to generate, this mean that a class will have a stereotype to identify if this
class will be generated or not, this is for not re-write classes already
generated. This objective is considered a critical factor for the success of
the project.

\item Alma Common Software - Notification Channels Support:\\ 
A full code generation support for ACS Notification Channels, generate all
necessary code from a UML model that represents the association between the
Supplier and Consumers, the generator must be able to generate the
ready-to-implement code for Notification Channels. This objective is
considered a critical factor for the success of the project.

\item Standalone Code Generator: \\
The development of the generator should be under the EMF, however one of the
critical requirements is have the generator as Standalone, this means that the
generator have to work outside the Eclipse IDE dinamicly with any UML
model/metamodel specified by command line. The generator have to generate all
the classes, plus empty skelletons for the user implementation. or regenerate
only the basic classes, this mean only generates the templates for user
modification. This objective is considered a critical factor for the success
of the project.

\item The generated code must be Java:\\ 
At first, all the code generated must be in Java, later, if the time allow,
then implement the code generator templates for C++ and Python, it's more
importante in this project have a full support Java for the code generated
than a semi-implementation of all programming languages.

\item Separation between the generated classes and implemented classes:\\
Complete the separation between the generated classes and the implemented
classes, this mean the code generator has to know how to separate the
implemented classes and the already generated classesfor the code generation
for not overwriting.

\item Inheritance implementation:\\
Actually, the implementation in the generator of model-inheritance is not fully
supported, the developers should define how implement this model behavior in
the code generator without affect all work already done.
\end{itemize}

\subsection{Project Scope}
The project scope cover the development of the Code Generator for ACS,
initially the project will only generate the Java code implementation from a
UML model under the XMI2.0 standard, in which, there are three importants
points defined as critical and project milestones.

\begin{itemize}
	\item Standalone Java Component Code Generator.
	\item Notification Channels and Inheritance Support for UML models.
	\item Code Generator based on Open Architecture Ware 5.
\end{itemize}
Each one of this points are explained more consistently in Goals and Objectives
sections in this document.

\subsection{Roles and Responsibilities}
\begin{itemize}
\item Alexis Tejeda (UCN) Deployment Manager, Requirements Reviewer,
Architecture Reviewer, Configuration Manager, Change Control Manager . 
\item Gianluca Chiozzi (ESO) Client, End User, Requirements Reviewer, Change
Control Manager.
\item Jorge Ibsen (ESO / JAO Computing Manager) Project Manager, Requirements
Reviewer.
\item Nicolas Troncoso (JAO/AUI) Project Manager, Requirements Reviewer,
Architecture Reviewer.
\end{itemize} 

\newpage









