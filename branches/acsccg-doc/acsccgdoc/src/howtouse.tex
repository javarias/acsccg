\section{HOW TO USE}

The usage of the code generator is pretty simple:

\begin{itemize}
\item Since the generator was designed to work better with MagicDraw models,
first create the model with the stereotypes defined in this document, a template
MagicDraw file can be downloded from:\\
http://acsccg.googlecode.com/files/TemplateModelMagicDraw.tar.gz .
\item It is recommended to create a folder with 3 subfolders inside, a
`generated` folder, a `model` folder and a `xmi` folder in which goes the
exported XMI files from MagicDraw.
\item Once the model is defined export the model in : `Eclipse UML2 (v2.x) XMI
FILE` export menu in MagicDraw.
\item Download the Jar Binary from:\\
http://acsccg.googlecode.com/files/ACSCCG.jar
\item Run the generator specifying the model path exported, profile file path
exported and the code path for the code generated:\\ 
\verb+java -jar ACSCCG.jar model-path profile-path output-path+
\item In the project code page, they are 6 examples of models, image models,
XMI, generated code (tested), ready to download and run, each example represents
a feature of the generator.
\item Update the make file, see the `Knowing Problems` section in this document.
\end{itemize} 